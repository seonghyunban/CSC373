    \section*{Q3}
    
    
    \begin{center}
    \begin{tikzpicture}[
      vertex/.style = {shape=circle,draw,minimum size=2.5em},
      edge/.style = {->,-Latex, thick},
      ]
    
    \node[vertex] (s) at (0,0) {$s$};
    
    \node[vertex] (u1) at (3,3) {$u_1$};
    \node (udots1) at (3,1.5) {\vdots};
    \node[vertex] (uk) at (3,0) {$u_k$};
    \node (udots2) at (3,-1.5) {\vdots};
    \node[vertex] (up) at (3,-3) {$u_p$};
    
    \node[vertex] (w1) at (7,3) {$w_1$};
    \node (wdots1) at (7,1.5) {\vdots};
    \node[vertex] (wl) at (7,0) {$w_l$};
    \node (wdots2) at (7,-1.5) {\vdots};
    \node[vertex] (wq) at (7,-3) {$w_q$};
    
    \node[vertex] (t) at (10,0) {$t$};
    
    \draw[edge] (s) -- (u1) node[midway, above, sloped] {$m_1$};
    \draw[edge] (s) -- (uk) node[midway, above, sloped] {$m_k$};
    \draw[edge] (s) -- (up) node[midway, above, sloped] {$m_p$};
    
    \foreach \u in {u1, uk, up} {
        \foreach \w in {w1, wl, wq} {
            \draw[edge] (\u) -- (\w) node[pos=0.1, above, sloped, font=\tiny] {$1$};
        }
    }
    
    \draw[edge] (w1) -- (t) node[midway, above, sloped] {$n_1$};
    \draw[edge] (wl) -- (t) node[midway, above, sloped] {$n_l$};
    \draw[edge] (wq) -- (t) node[midway, above, sloped] {$n_p$};
    
    % Cut A
    \begin{scope}[on background layer]
    \draw[thick, dashed, blue, rounded corners=8pt, fill=cyan!20, opacity=0.4]
      ($(s) + (-0.6,-0.6)$) -- ($(s) + (-0.6, 0.6)$) 
      -- ($(u1) + (-0.6, 0.6)$) -- ($(w1) + (0.6, 0.6)$) 
      -- ($(wl) + (0.6,-0.6)$) -- ($(uk) + (-0.6,-0.6)$) 
      -- cycle;
    \end{scope}
    
    \node[anchor=south east] at ($(s) + (-0.5,1.5)$) {\textbf{Cut A}};
    
    % Cut B
    \begin{scope}[on background layer]
    \draw[thick, dashed, red, rounded corners=8pt, fill=yellow!20, opacity=0.4]
      ($(up) + (-0.6,-0.6)$) -- ($(udots2) + (-0.6, 0.6)$) % edge around up
      -- ($(wdots2) + (0.6, 0.6)$) -- ($(t) + (-0.6, 0.6)$)
      -- ($(t) + (0.6,0.6)$)
      -- ($(t) + (0.6,-0.6)$) -- ($(wq) + (0.6,-0.6)$) 
      -- ($(up) + (0.6,-0.6)$) 
      -- cycle;
    \end{scope}
    
    \node[anchor=south east] at ($(t) + (1.5,-1.5)$) {\textbf{Cut B}};
    
    \end{tikzpicture}
    \end{center}
    
    
    \subsection*{(a)}
    
    Consider this flow network, where $u$ nodes represent universities and $w$ nodes represent workshops.
    
    The source can send to a university, all its students $m_i$. Since the capacity between each university and workshop node is $1$, only one university from each university can go to a given workshop. Since the capacities from the workshops to the sink is the workshop's capacities $n_j$, that puts a limit on how many students can enter a given workshop. Finding the maximum flow of this graph tells us the optimal dispatching of participants into workshops, satisfying these constraints.
    
    \subsection*{(b)}
    
    Let cut $A$ and $B = V \setminus A$ be defined as follows:
    \begin{enumerate}
        \item $A = \{s, u_1, \dots, u_k, w_1, \dots, w_l\}$
        \item $B = \{t, u_{k+1}, \dots, u_p, w_{l+1}, \dots, w_q\}$
    \end{enumerate}
    
    WTS
    
    $\exists$ a valid satisfying dispatching $\iff$ $\forall k \in {0, \dots, p}, \forall l \in {0, \dots, q}, k(q-l) + \sum_{j=1}^{l} n_k \geq \sum_{i=1}^{k} m_i$.
    
    We will prove this in both directions.
    
    
    
    \subsubsection*{(Backward $\impliedby$)}