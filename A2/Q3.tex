\section*{Q3}


\begin{center}
\begin{tikzpicture}[
  vertex/.style = {shape=circle,draw,minimum size=2.5em},
  edge/.style = {->,-Latex, thick},
  ]

% Setup nodes
\node[vertex] (s) at (0,0) {$s$};

\node[vertex] (u1) at (3,3) {$u_1$};
\node (udots1) at (3,1.5) {\vdots};
\node[vertex] (uk) at (3,0) {$u_k$};
\node (udots2) at (3,-1.5) {\vdots};
\node[vertex] (up) at (3,-3) {$u_p$};

\node[vertex] (w1) at (7,3) {$w_1$};
\node (wdots1) at (7,1.5) {\vdots};
\node[vertex] (wl) at (7,0) {$w_l$};
\node (wdots2) at (7,-1.5) {\vdots};
\node[vertex] (wq) at (7,-3) {$w_q$};

\node[vertex] (t) at (10,0) {$t$};

% From source to universities
\draw[edge] (s) -- (u1) node[midway, above, sloped] {$m_1$};
\draw[edge] (s) -- (uk) node[midway, above, sloped] {$m_k$};
\draw[edge, color=magenta] (s) -- (up) node[midway, above, sloped] {$m_p$};

\foreach \u in {u1, uk, up} {
    \foreach \w in {w1, wl, wq} {
        \draw[edge] (\u) -- (\w) node[pos=0.1, above, sloped, font=\tiny] {$1$};
    }
}

% Override color
\draw[edge, color=magenta] (u1) -- (wq) node[pos=0.1, above, sloped, font=\tiny] {$1$};
\draw[edge, color=magenta] (uk) -- (wq) node[pos=0.1, above, sloped, font=\tiny] {$1$};

% From workshops to sink
\draw[edge, color=magenta] (w1) -- (t) node[midway, above, sloped] {$n_1$};
\draw[edge, color=magenta] (wl) -- (t) node[midway, above, sloped] {$n_l$};
\draw[edge] (wq) -- (t) node[midway, above, sloped] {$n_p$};

% Cut A
\begin{scope}[on background layer]
\draw[thick, dashed, blue, rounded corners=8pt, fill=cyan!20, opacity=0.4]
  ($(s) + (-0.6,-0.6)$) -- ($(s) + (-0.6, 0.6)$) 
  -- ($(u1) + (-0.6, 0.6)$) -- ($(w1) + (0.6, 0.6)$) 
  -- ($(wl) + (0.6,-0.6)$) -- ($(uk) + (-0.6,-0.6)$) 
  -- cycle;
\end{scope}

\node[anchor=south east] at ($(s) + (-0.5,1.5)$) {\textbf{Cut A}};

% Cut B
\begin{scope}[on background layer]
\draw[thick, dashed, red, rounded corners=8pt, fill=yellow!20, opacity=0.4]
  ($(up) + (-0.6,-0.6)$) -- ($(udots2) + (-0.6, 0.6)$) % edge around up
  -- ($(wdots2) + (0.6, 0.6)$) -- ($(t) + (-0.6, 0.6)$)
  -- ($(t) + (0.6,0.6)$)
  -- ($(t) + (0.6,-0.6)$) -- ($(wq) + (0.6,-0.6)$) 
  -- ($(up) + (0.6,-0.6)$) 
  -- cycle;
\end{scope}

\node[anchor=south east] at ($(t) + (1.5,-1.5)$) {\textbf{Cut B}};

\end{tikzpicture}
\end{center}


\subsection*{(a)}

Consider this flow network, where $u$ nodes represent universities and $w$ nodes represent workshops.

The source can send to a university, all its students $m_i$. Since the capacity between each university and workshop node is $1$, only one university from each university can go to a given workshop. Since the capacities from the workshops to the sink is the workshop's capacities $n_j$, that puts a limit on how many students can enter a given workshop. Finding the maximum flow of this graph tells us the optimal dispatching of participants into workshops, satisfying these constraints.

\subsection*{(b)}

Let cut $A$ and $B = V \setminus A$ be an arbitrary cut defined by $k$ and $l$:
\begin{enumerate}
    \item $A = \{s, u_1, \dots, u_k, w_1, \dots, w_l\}$
    \item $B = \{t, u_{k+1}, \dots, u_p, w_{l+1}, \dots, w_q\}$
\end{enumerate}

The capacity of cut A is the flow leaving it (purple arrows in diagram). This gives:

$cap(A,B) = k(q-l) + \sum_{j=1}^{l}n_j + \sum_{i=k+1}^{p} m_i$

\begin{enumerate}
    \item $\sum_{i=k+1}^{p} m_i$ represents the edges from the source to universities in $A$
    \item $\sum_{j=1}^{l}n_j$ represents the edges from workshops in $B$ to the sink
    \item $k(q-l)$ represents the edges from univerisites in $A$ to workshops in $B$
\end{enumerate}

WTS $\exists$ a valid dispatching $\iff$ $\forall k \in \{0, \dots, p\}, \forall l \in \{0, \dots, q\}, k(q-l) + \sum_{j=1}^{l} n_k \geq \sum_{i=1}^{k} m_i$.

\subsubsection*{Forward Proof ($\implies$):}
Assume we have a valid dispatching of students satisfying the constraints.

We know the minimum capacity cut must have a flow of at least the number of students we need to send. If a single cut had less capacity than the flow, it would violate our assumption. So for all possible cuts:

$cap(A,B) \geq v(f) = \sum_{i=1}^{p} m_i$

$k(q-l) + \sum_{j=1}^{l}n_j + \sum_{i=k+1}^{p} m_i \geq \sum_{i=1}^{p} m_i$

$k(q-l) + \sum_{j=1}^{l}n_j + \sum_{i=k+1}^{p} m_i \geq \sum_{i=1}^{k} m_i + \sum_{i=k+1}^{p} m_i$

$k(q-l) + \sum_{j=1}^{l}n_j +\geq \sum_{i=1}^{k} m_i$

\subsubsection*{Backward Proof ($\impliedby$):}
In the previous part, we showed that $\exists \text{flow} f, \iff \text{valid assignment}$

Assume $\forall k \in \{0, \dots, p\}, \forall l \in \{0, \dots, q\}, k(q-l) + \sum_{j=1}^{l} n_k \geq \sum_{i=1}^{k} m_i$

From the same steps in the previous problem done backwards, we have:

By varying $k$ and $l$, all cuts $A$ and $B$ in flow network $G$ can be created.

$\forall k \in \{0, \dots, p\}, \forall l \in \{0, \dots, q\}, cap(A,B) \geq v(f) = \sum_{i=1}^{p} m_i$

Since every cut has $cap(A,B) \geq \sum_{i=1}^{p} m_i$, we know the minimum capacity cut also has this property.

Therefore, there $\exists \text{flow} f | v(f) = \sum_{i=1}^{p} m_i$. Run the FF algorithm on the flow network defined in part A, then we get a valid dispatching.
