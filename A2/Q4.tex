\section*{Q4}

\subsection*{(a)}
\begin{center}
\begin{tikzpicture}[
  vertex/.style = {shape=circle,draw,minimum size=2.5em, font=\Large},
  edge/.style = {->,-Latex, thick},
  ]

\node[vertex] (s) at (0,0) {$s$};
\node[vertex] (a) at (3,0) {$a$};
\node[vertex] (b) at (6,0) {$b$};
\node[vertex] (t) at (9,0) {$t$};

\draw[edge] (s) -- (a);
\draw[edge] (b) -- (t);

\draw[edge] (a) to[bend left=20] node[midway, above] {$e$} (b);
\draw[edge] (b) to[bend left=20] node[midway, below] {$e'$} (a);

\end{tikzpicture}
\end{center}

Let $a,b$ be nodes such that $e = (a,b)$ and $e' = (b,a)$. let $A$ be the min cut and $B = V \setminus A$. We don't put any restrictions on the graph outside of that.

Suppose there is a maximum flow in which $e$ and $e'$ both have non zero flow: $f(e) > 0 \wedge f(e') > 0$

Let $\Delta = min(f(e),f(e'))$ and $g(e)=f(e)-\Delta$ and $g(e')=f(e')-\Delta$

The net flow remains the same: $g(e)-g(e') = (f(e)-\Delta) - (f(e')-\Delta) = f(e)-f(e')$

If $f(e)>f(e')$, then $\Delta=f(e')$, so $g(e)=f(e)-f(e')>0$ and $g(e')=f(e')-f(e')=0$

If $f(e')>f(e)$, then $\Delta=f(e)$, so $g(e')=f(e')-f(e)>0$ and $g(e)=f(e)-f(e)=0$

We have shown that for every maximum flow, there is a corresponding max flow where atleast one of the edges has zero flow.

\subsection*{(b)}
The algorithm works as follows:
Create a new graph $G'$ which sets each edge of $G$ to 1.

Run the FF algorithm to find the max flow and its corresponding min cut, $A$

The FF algorithm finds the minimum capacity cut who's capacity is equal to its flow ($Cap(A) = v(f)$). So each unit of flow represents a unique edge crossing the min cut, since all edge weights are 1. Removing these edges disconnects the two cuts, and since this is the minimum capacity cut that achieves this, we know this is the minimum number of edges to be removed from $G'$ so that no flow passes through from $s$ to $t$, or equivalently, there is no path from $s$ to $t$ in $G$.