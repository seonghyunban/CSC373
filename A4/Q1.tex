\section*{Q1}


\subsection*{(a)}

Suppose we start from a maximal assignment $M'$, and try to construct a new assignment $M$ with more groups. This is always possible by breaking all existing groups in $M'$ and forming new groups from scratch.

Since $M'$ is maximal, any new group we form in $M$ must overlap with some group in $M'$ — otherwise, that group could have been added to $M'$ without breaking existing groups, contradicting maximality.

Furthermore, each group in $M'$ can contribute to at most 3 groups in $M$, because forming more than 3 groups overlapping with the same group in $M'$ would force overlap between groups in $M$, violating disjointness.

Thus, for any maximal assignment $M$, we must have:
\[
|M| \leq 3|M'|.
\]

\subsection*{(b)}

Define $N(v)$ as the set of neighbors of $v$ in $G$.

\begin{verbatim}
Algorithm GreedyMaximalAssignment(G = (V, E)):
    M ← {}
    While V is not empty:
        let v be a vertex in V
        grouped ← False
        for each u in the intersection of N(v) and V:
            if not grouped:
            for each w inthe intersection of N(u), N(v) and V:
                    Add {u, v, w} to M
                    Remove {u, v, w} from V
                    grouped ← True
        if not grouped:
            V ← V - {v}
    return M
\end{verbatim}

At the end, $M$ is maximal because we cannot insert any new group of three compatible students (i.e., a 3-clique) into the assignment.

Let $M^*$ be an optimal (maximum) assignment. By part~(a), we have:
\[
|M| \ge \tfrac{1}{3} |M^*|,
\]
so $M$ is a 3-approximation.

In the worst case (dense graphs), This algorithm runtime is $O(n^3)$.