\section*{Q2}

\subsection*{(a)}

\begin{verbatim}

MinimizeSumFinishTime(J):

    J = list of jobs [(s, p, rem, k) for i in 1, ... , n] sorted by release time s.
    # For each tuple in J, k is the index and rem is the remaining time for the job.
    
    P = priority queue ordered by remaining time (rem)
    i = 0                
    t = 0                
    sum_fin = 0
    order = []

    while i < len(J) or not P.is_empty():
        
        # Enqueue all jobs starting before or at t
        while i < len(J) and J[i].s <= t:
            P.enqueue(J[i])
            i += 1
    
        # If no jobs are available yet, skip to next start time
        if P is empty:
            t = J[i].s
            continue
    
        # Get job with shortest remaining time
        curr = P.dequeue()
    
        # Determine available time until next release
        if i < len(J):
            available = J[i].s - t
        else:
            available = infinty
    
        run_time = min(curr.rem, available)
        curr.rem -= run_time
        t += run_time
    
        if curr.rem == 0:
            sum_finish += t
            order.append(curr.k)
        else:
            P.enqueue(curr)
    
    return sum_finish, order
    \end{verbatim}

\newpage

\subsection*{(b)}

\begin{verbatim}

ApproximateMinimizeSumFinishTime(J):

    _, order = MinimizeSumFinishTime(J)

    J = list of jobs [(s, p, rem, k) for i in 1, ... , n] sorted by release time s.
    J = reorder J according to order

    t = 0
    sum_fin = 0

    for each job in J:
        if job.s > t:
            t = job.s
        t += job.p
        sum_fin += t

    return sum_fin
\end{verbatim}


\textbf{Claim.}
 This schedule is a 2-approximation.

\vspace{0.5em}
\textbf{Proof.}

Define:
\begin{itemize}
    \item $F_j^{\text{SRPT}}$ = finish time of job $j$ in the shortest remaining processing time schedule,
    \item $F_j^{\text{NEW}}$ = finish time of job $j$ in the nonpreemptive schedule that orders jobs by increasing $F_j^{\text{SRPT}}$,
    \item $B_j^{\text{NEW}}$ = total busy (processing) time of the machine on jobs $1$ to $j$ in nonpreemptive algorithm,
    \item $I_j^{\text{NEW}}$ = total idle time before finishing job $j$ in nonpreemptive algorithm.
\end{itemize}

Call the nonpreemptive algorithm as NEW.

We decompose:
\[
F_j^{\text{NEW}} = B_j^{\text{NEW}} + I_j^{\text{NEW}}.
\]

Since $B_j^{\text{NEW}} = \sum_{i=1}^j p_i$, and SRPT finishes jobs $1$ through $j$ by time $F_j^{\text{SRPT}}$, we have trivially:
\[
B_j^{\text{NEW}} \le B_j^{\text{SRPT}} \le F_j^{\text{SRPT}}.
\]

Idle time in NEW occurs only if a job $i$ is scheduled but has not yet been released at the time it is ready to be started. Let $F_0^{\text{NEW}} := 0$, then:
\[
I_j^{\text{NEW}} = \sum_{i=1}^{j} (s_i - F_{i-1}^{\text{NEW}}) \cdot \mathbf{I}[s_i > F_{i-1}^{\text{NEW}}].
\]

Using the fact that $F_{i-1}^{\text{NEW}} \ge F_{i-1}^{\text{SRPT}}$ for all $i$ (since SRPT is optimal per job):
\[
I_j^{\text{NEW}} \le \sum_{i=1}^{j} (s_i - F_{i-1}^{\text{SRPT}}) \cdot \mathbf{I}[s_i > F_{i-1}^{\text{SRPT}}] = I_j^{\text{SRPT}}.
\]

Finally, since $I_j^{\text{SRPT}} = F_j^{\text{SRPT}} - B_j^{\text{SRPT}} \le F_j^{\text{SRPT}}$, we conclude:
\[
I_j^{\text{NEW}} \le F_j^{\text{SRPT}}.
\]

Combining the inequalities:
\[
F_j^{\text{NEW}} = B_j^{\text{NEW}} + I_j^{\text{NEW}} \le F_j^{\text{SRPT}} + F_j^{\text{SRPT}} = 2 F_j^{\text{SRPT}}.
\]

\hfill$\blacksquare$

    
