\section*{Q2}

\subsection*{(a)}

\begin{verbatim}

MinimizeSumFinishTime(J):

    J = list of jobs [(s, p, rem, k) for i in 1, ... , n] sorted by release time s.
    # For each tuple in J, k is the index and rem is the remaining time for the job.
    
    P = priority queue ordered by remaining time (rem)
    i = 0                
    t = 0                
    sum_fin = 0
    order = []

    while i < len(J) or not P.is_empty():
        
        # Enqueue all jobs starting before or at t
        while i < len(J) and J[i].s <= t:
            P.enqueue(J[i])
            i += 1
    
        # If no jobs are available yet, skip to next start time
        if P is empty:
            t = J[i].s
            continue
    
        # Get job with shortest remaining time
        curr = P.dequeue()
    
        # Determine available time until next release
        if i < len(J):
            available = J[i].s - t
        else:
            available = infinty
    
        run_time = min(curr.rem, available)
        curr.rem -= run_time
        t += run_time
    
        if curr.rem == 0:
            sum_finish += t
            order.append(curr.k)
        else:
            P.enqueue(curr)
    
    return sum_finish, order
    \end{verbatim}

\newpage

\subsection*{(b)}

\begin{verbatim}

ApproximateMinimizeSumFinishTime(J):

    _, order = MinimizeSumFinishTime(J)

    J = list of jobs [(s, p, rem, k) for i in 1, ... , n] sorted by release time s.
    J = reorder J according to order

    t = 0
    sum_fin = 0

    for each job in J:
        if job.s > t:
            t = job.s
        t += job.p
        sum_fin += t

    return sum_fin
\end{verbatim}

\textbf{Justification}:
...
    
