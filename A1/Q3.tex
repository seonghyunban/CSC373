\section*{Q3}

\begin{algorithm}[H]
\caption{3 Split Huffman Encoding}  
\begin{algorithmic}[1]
\vspace{1em}

\Function{Huffman}{$S$} \Comment{F is the prioroty queue of (frequency, letter) where frequency is key}

    \vspace{1em}
    \If {size of $S \leq 3$} \Comment{Base case returns correct encoding when size $\leq$ 3}
        \State \Return a single node tree if $|S| = 1$.
        \State \Return a tree with two leaves if $|S| = 2$.
        \State \Return a tree with three leaves if $|S| = 3$.
    \EndIf

    \vspace{1em}

    \vspace{1em}
    \State If the size of $S$ is even pad $S$ with a dummy character with frequency = 0.

    \vspace{1em}

    \State $x, y, z \gets $ three smallest frequency elements in $S$.
    \State Let $w := 'xyz'$ 
    \State Let $f_w := f_x + f_y + f_z$
    \State Push $(f_w, w)$ to $S$ \Comment{Create input with smaller size for recursive call}
    \State $H \gets \textbf{Huffman}(S)$.

    \vspace{1em}
    \State Find the node $w^*$ in $H$ that corresponds to $w$.
    \State \textbf{Add} branches as children of $w^*$ . \Comment{This assigns encoding to branches.}
    \State \Return $H$

    \vspace{1em}
    \EndFunction

\vspace{1em}
\end{algorithmic}
\end{algorithm}

\subsection*{Time Complexity}
The base case takes $O(1)$ time.
For the recursive  case, popping the three or two smallest elements takes $O(\log n)$ time.
All operations before the recursive call take $O(1)$ time, 
and all operations after the recursive call take $O(1)$ time 
(e.g., the tree is implemented using direct access table).
Since branching by factor of 3 will reduce the time complexity more than by branching factor of 2, 
The time complexity of this algorithm is given by the recurrence relation $T(n) \leq T(n - 1) + O(\log n)$ = recurrence relation of original Huffman algorithm.
From lecture, original Huffman algorithm takes $O(n\log n)$, thus, $T(n) \in O(n \log n)$.


\subsection*{Correctness}

\textbf{(Lemma 1)} In an optimal ternary tree that is not a single node tree, every node other than the root must have a sibling, because if otherwise, 
replacing the parent of a single child with its child gives tree with lower total length, contradicting that the original tree was optimal. 

\textbf{(Lemma 2)}
The optimal ternary tree $T$ is a full ternary tree if it has odd number of leaves $\geq 3$. Suppose for contradiction that it is not. 
Then there is at least one node that has binary branching, let $v$ be the deepest such node.
If $v$ is not a parent of two leaves, 
by moving a non-child leaf of the subtree rooted at $v$ 
as the child of $v$
we can find a tree with lower loss than $T$.
If $v$ is a parent of two leaves, there are two cases. 
If there is another binary branching at node $u$, 
moving a child of $v$ as a child of $u$ and replacing $v$ with its remaining child 
makes a lower cost tree.
If there no other binary branching, then the tree cannot have odd number of leaves, 
since for each node from $v$ to root, it give rise to $3^k_1 + 3^k_2$ leaves for some $k_1, k_2$, and this number must be even.


\textbf{(Lemma 3)} In optimal ternary tree, the  three lowest freqeuncy nodes $x, y, z$ (in this order) must be the three deepest node (in that order).
Suppose for contradiction that it is not. Let $p \in {x, y, z}$ then there exists at least one $q$ such that $q$ is more frequent than $p$ but located deeper than $p$.
swapping $p$ and $q$ leads to a tree with lower total length, which is a contradiction.\\


\textbf{Proof.}\\

As the base case, when the size of $|S| \leq 3$ the algorithm outputs optimal tree.

In any other cases the algorithm ensure that $|S|$ is odd at line 7.

Assume as IH that algotithm outputs optimal tree when size of input is less than $k \in \NN^{\geq 3}$.

Let $S$ be an input of size $k$. 
Let $H$ be the output of the algorithm given $S$.
Let $S'$ be the $S$ after operations at line 8 to 11.
Then the recursive call at line 12 returns the optimal tree, $H'$ over $S'$ by IH.

Let $f_w$ be the sum of frequencies of $x, y, z$.
We know that total loss of S, $loss(H) = f_w + loss(H')$ 
(using the same reasoning as KT p.174 but using three frequencies instead of two).

Suppose for contradiction $H$ is not optimal.
Then there exists an optimal tree $T$ over $S$, and thus $loss(H) > loss(T)$.
Because $T$ is optimal, the three lowest frequency nodes appear as siblings at the deepest level of T.
(by lemma 2 and 3).
Let $T'$ be a tree after removing the three lowest frequency nodes from T. 
Similarly as above, $T'$ is a optimal tree over $S'$ and we can show that 
$loss(T) = f_w + loss(T')$ by using the same reasoning as above.

By IH, $loss(H') \leq loss(T')$. However, then $loss(H) = f_w + loss(H') \leq f_w + loss(T') = loss(T)$, which is a contradiction. \qed




